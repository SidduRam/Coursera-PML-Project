\documentclass[]{article}
\usepackage[T1]{fontenc}
\usepackage{lmodern}
\usepackage{amssymb,amsmath}
\usepackage{ifxetex,ifluatex}
\usepackage{fixltx2e} % provides \textsubscript
% Set line spacing
% use upquote if available, for straight quotes in verbatim environments
\IfFileExists{upquote.sty}{\usepackage{upquote}}{}
\ifnum 0\ifxetex 1\fi\ifluatex 1\fi=0 % if pdftex
  \usepackage[utf8]{inputenc}
\else % if luatex or xelatex
  \ifxetex
    \usepackage{mathspec}
    \usepackage{xltxtra,xunicode}
  \else
    \usepackage{fontspec}
  \fi
  \defaultfontfeatures{Mapping=tex-text,Scale=MatchLowercase}
  \newcommand{\euro}{€}
\fi
% use microtype if available
\IfFileExists{microtype.sty}{\usepackage{microtype}}{}
\usepackage[margin=1in]{geometry}
\usepackage{color}
\usepackage{fancyvrb}
\newcommand{\VerbBar}{|}
\newcommand{\VERB}{\Verb[commandchars=\\\{\}]}
\DefineVerbatimEnvironment{Highlighting}{Verbatim}{commandchars=\\\{\}}
% Add ',fontsize=\small' for more characters per line
\usepackage{framed}
\definecolor{shadecolor}{RGB}{248,248,248}
\newenvironment{Shaded}{\begin{snugshade}}{\end{snugshade}}
\newcommand{\KeywordTok}[1]{\textcolor[rgb]{0.13,0.29,0.53}{\textbf{{#1}}}}
\newcommand{\DataTypeTok}[1]{\textcolor[rgb]{0.13,0.29,0.53}{{#1}}}
\newcommand{\DecValTok}[1]{\textcolor[rgb]{0.00,0.00,0.81}{{#1}}}
\newcommand{\BaseNTok}[1]{\textcolor[rgb]{0.00,0.00,0.81}{{#1}}}
\newcommand{\FloatTok}[1]{\textcolor[rgb]{0.00,0.00,0.81}{{#1}}}
\newcommand{\CharTok}[1]{\textcolor[rgb]{0.31,0.60,0.02}{{#1}}}
\newcommand{\StringTok}[1]{\textcolor[rgb]{0.31,0.60,0.02}{{#1}}}
\newcommand{\CommentTok}[1]{\textcolor[rgb]{0.56,0.35,0.01}{\textit{{#1}}}}
\newcommand{\OtherTok}[1]{\textcolor[rgb]{0.56,0.35,0.01}{{#1}}}
\newcommand{\AlertTok}[1]{\textcolor[rgb]{0.94,0.16,0.16}{{#1}}}
\newcommand{\FunctionTok}[1]{\textcolor[rgb]{0.00,0.00,0.00}{{#1}}}
\newcommand{\RegionMarkerTok}[1]{{#1}}
\newcommand{\ErrorTok}[1]{\textbf{{#1}}}
\newcommand{\NormalTok}[1]{{#1}}
\ifxetex
  \usepackage[setpagesize=false, % page size defined by xetex
              unicode=false, % unicode breaks when used with xetex
              xetex]{hyperref}
\else
  \usepackage[unicode=true]{hyperref}
\fi
\hypersetup{breaklinks=true,
            bookmarks=true,
            pdfauthor={},
            pdftitle={Practical Machine Learning Assignment},
            colorlinks=true,
            citecolor=blue,
            urlcolor=blue,
            linkcolor=magenta,
            pdfborder={0 0 0}}
\urlstyle{same}  % don't use monospace font for urls
\setlength{\parindent}{0pt}
\setlength{\parskip}{6pt plus 2pt minus 1pt}
\setlength{\emergencystretch}{3em}  % prevent overfull lines
\setcounter{secnumdepth}{0}

%%% Change title format to be more compact
\usepackage{titling}
\setlength{\droptitle}{-2em}
  \title{Practical Machine Learning Assignment}
  \pretitle{\vspace{\droptitle}\centering\huge}
  \posttitle{\par}
  \author{}
  \preauthor{}\postauthor{}
  \date{}
  \predate{}\postdate{}




\begin{document}

\maketitle


\subsection{Executive summary}\label{executive-summary}

Analyze the device data from Jawbone Up, Nike FuelBand, and Fitbit for 6
participants, from their ccelerometers on the belt, forearm, arm, and
dumbelldata for 5 dataset using linear regression models. This learning
will quantify how much of a particular activity they do, but they rarely
quantify how well they do it.

\subsection{Reading the data}\label{reading-the-data}

\subsubsection{The Raw Data - Download the file if does not exist in
local
system}\label{the-raw-data---download-the-file-if-does-not-exist-in-local-system}

\begin{Shaded}
\begin{Highlighting}[]
\NormalTok{if (!}\KeywordTok{file.exists}\NormalTok{(}\StringTok{"./Data/pml-training.csv"}\NormalTok{))\{}
  \KeywordTok{download.file}\NormalTok{(}\StringTok{"https://d396qusza40orc.cloudfront.net/predmachlearn/pml-training.csv"}\NormalTok{,}
                \StringTok{"./Data/pml-training.csv"}\NormalTok{)}
\NormalTok{\}}
\NormalTok{if (!}\KeywordTok{file.exists}\NormalTok{(}\StringTok{"./Data/pml-testing.csv"}\NormalTok{))\{}
  \KeywordTok{download.file}\NormalTok{(}\StringTok{"https://d396qusza40orc.cloudfront.net/predmachlearn/pml-training.csv"}\NormalTok{,}
                \StringTok{"./Data/pml-testing.csv"}\NormalTok{)}
\NormalTok{\}}
\end{Highlighting}
\end{Shaded}

\subsubsection{Load the training and testing
data}\label{load-the-training-and-testing-data}

\begin{Shaded}
\begin{Highlighting}[]
\NormalTok{trainingdata =}\StringTok{ }\KeywordTok{read.csv}\NormalTok{(}\StringTok{"./Data/pml-training.csv"}\NormalTok{, }\DataTypeTok{na.strings =} \KeywordTok{c}\NormalTok{(}\StringTok{"NA"}\NormalTok{, }\StringTok{""}\NormalTok{))}
\KeywordTok{dim}\NormalTok{(trainingdata); }\KeywordTok{summary}\NormalTok{(trainingdata$classe)}
\end{Highlighting}
\end{Shaded}

\begin{verbatim}
## [1] 19622   160
\end{verbatim}

\begin{verbatim}
##    A    B    C    D    E 
## 5580 3797 3422 3216 3607
\end{verbatim}

\begin{Shaded}
\begin{Highlighting}[]
\NormalTok{testingdata =}\StringTok{ }\KeywordTok{read.csv}\NormalTok{(}\StringTok{"./Data/pml-testing.csv"}\NormalTok{, }\DataTypeTok{na.strings =} \KeywordTok{c}\NormalTok{(}\StringTok{"NA"}\NormalTok{, }\StringTok{""}\NormalTok{))}
\end{Highlighting}
\end{Shaded}

\subsubsection{Load the library}\label{load-the-library}

\begin{Shaded}
\begin{Highlighting}[]
\KeywordTok{library}\NormalTok{(ggplot2); }\KeywordTok{library}\NormalTok{(caret);}\KeywordTok{library}\NormalTok{(randomForest)}
\end{Highlighting}
\end{Shaded}

\subsubsection{Removing nerar Zero
covariates}\label{removing-nerar-zero-covariates}

\begin{Shaded}
\begin{Highlighting}[]
\NormalTok{nzv <-}\StringTok{ }\KeywordTok{nearZeroVar}\NormalTok{(trainingdata,}\DataTypeTok{saveMetrics=}\OtherTok{TRUE}\NormalTok{)}
\NormalTok{trainingdata <-}\StringTok{ }\NormalTok{trainingdata[,nzv$nzv==}\OtherTok{FALSE}\NormalTok{]}

\NormalTok{nzv <-}\StringTok{ }\KeywordTok{nearZeroVar}\NormalTok{(testingdata,}\DataTypeTok{saveMetrics=}\OtherTok{TRUE}\NormalTok{)}
\NormalTok{testingdata <-}\StringTok{ }\NormalTok{testingdata[,nzv$nzv==}\OtherTok{FALSE}\NormalTok{]}
\end{Highlighting}
\end{Shaded}

\subsection{Partioning the training
datset}\label{partioning-the-training-datset}

\begin{Shaded}
\begin{Highlighting}[]
\NormalTok{inTrain <-}\StringTok{ }\KeywordTok{createDataPartition}\NormalTok{(}\DataTypeTok{y=}\NormalTok{trainingdata$classe, }\DataTypeTok{p=}\FloatTok{0.6}\NormalTok{, }\DataTypeTok{list=}\OtherTok{FALSE}\NormalTok{)}
\NormalTok{projTraining <-}\StringTok{ }\NormalTok{trainingdata[inTrain, ]; projTesting <-}\StringTok{ }\NormalTok{trainingdata[-inTrain, ]}
\KeywordTok{dim}\NormalTok{(projTraining); }\KeywordTok{dim}\NormalTok{(projTesting)}
\end{Highlighting}
\end{Shaded}

\begin{verbatim}
## [1] 11776   117
\end{verbatim}

\begin{verbatim}
## [1] 7846  117
\end{verbatim}

\subsubsection{Killing first column of Dataset(ID Removing first ID
variable) so that it does not interfer with ML
Algorithms.}\label{killing-first-column-of-datasetid-removing-first-id-variable-so-that-it-does-not-interfer-with-ml-algorithms.}

\begin{Shaded}
\begin{Highlighting}[]
\NormalTok{projTraining <-}\StringTok{ }\NormalTok{projTraining[}\KeywordTok{c}\NormalTok{(-}\DecValTok{1}\NormalTok{)]}
\end{Highlighting}
\end{Shaded}

\subsubsection{Remove the columns / Variables has too many NAs (keep
only the variable \textgreater{} 60\% threshold of
NA's)}\label{remove-the-columns-variables-has-too-many-nas-keep-only-the-variable-60-threshold-of-nas}

\begin{Shaded}
\begin{Highlighting}[]
\NormalTok{subprojTraining <-}\StringTok{ }\NormalTok{projTraining }
\NormalTok{for(i in }\DecValTok{1}\NormalTok{:}\KeywordTok{length}\NormalTok{(projTraining)) \{ }
  \NormalTok{if( }\KeywordTok{sum}\NormalTok{( }\KeywordTok{is.na}\NormalTok{( projTraining[, i] ) ) /}\KeywordTok{nrow}\NormalTok{(projTraining) >=}\StringTok{ }\NormalTok{.}\DecValTok{6} \NormalTok{) \{ }
    \NormalTok{for(j in }\DecValTok{1}\NormalTok{:}\KeywordTok{length}\NormalTok{(subprojTraining)) \{}
      \NormalTok{if( }\KeywordTok{length}\NormalTok{( }\KeywordTok{grep}\NormalTok{(}\KeywordTok{names}\NormalTok{(projTraining[i]), }\KeywordTok{names}\NormalTok{(subprojTraining)[j]) ) ==}\DecValTok{1}\NormalTok{)  \{ }
        \NormalTok{subprojTraining <-}\StringTok{ }\NormalTok{subprojTraining[ , -j] }
      \NormalTok{\}   }
    \NormalTok{\} }
  \NormalTok{\}}
\NormalTok{\}}

\CommentTok{#To check the new NA's of observations}
\KeywordTok{dim}\NormalTok{(subprojTraining); }\KeywordTok{str}\NormalTok{(subprojTraining)}
\end{Highlighting}
\end{Shaded}

\begin{verbatim}
## [1] 11776    58
\end{verbatim}

\begin{verbatim}
## 'data.frame':    11776 obs. of  58 variables:
##  $ user_name           : Factor w/ 6 levels "adelmo","carlitos",..: 2 2 2 2 2 2 2 2 2 2 ...
##  $ raw_timestamp_part_1: int  1323084231 1323084231 1323084232 1323084232 1323084232 1323084232 1323084232 1323084232 1323084232 1323084232 ...
##  $ raw_timestamp_part_2: int  788290 820366 304277 440390 484323 500302 528316 560359 576390 604281 ...
##  $ cvtd_timestamp      : Factor w/ 20 levels "02/12/2011 13:32",..: 9 9 9 9 9 9 9 9 9 9 ...
##  $ num_window          : int  11 11 12 12 12 12 12 12 12 12 ...
##  $ roll_belt           : num  1.41 1.42 1.45 1.42 1.43 1.45 1.43 1.42 1.42 1.45 ...
##  $ pitch_belt          : num  8.07 8.07 8.06 8.13 8.16 8.18 8.18 8.2 8.21 8.2 ...
##  $ yaw_belt            : num  -94.4 -94.4 -94.4 -94.4 -94.4 -94.4 -94.4 -94.4 -94.4 -94.4 ...
##  $ total_accel_belt    : int  3 3 3 3 3 3 3 3 3 3 ...
##  $ gyros_belt_x        : num  0 0 0.02 0.02 0.02 0.03 0.02 0.02 0.02 0 ...
##  $ gyros_belt_y        : num  0 0 0 0 0 0 0 0 0 0 ...
##  $ gyros_belt_z        : num  -0.02 -0.02 -0.02 -0.02 -0.02 -0.02 -0.02 0 -0.02 0 ...
##  $ accel_belt_x        : int  -21 -20 -21 -22 -20 -21 -22 -22 -22 -21 ...
##  $ accel_belt_y        : int  4 5 4 4 2 2 2 4 4 2 ...
##  $ accel_belt_z        : int  22 23 21 21 24 23 23 21 21 22 ...
##  $ magnet_belt_x       : int  -3 -2 0 -2 1 -5 -2 -3 -8 -1 ...
##  $ magnet_belt_y       : int  599 600 603 603 602 596 602 606 598 597 ...
##  $ magnet_belt_z       : int  -313 -305 -312 -313 -312 -317 -319 -309 -310 -310 ...
##  $ roll_arm            : num  -128 -128 -128 -128 -128 -128 -128 -128 -128 -129 ...
##  $ pitch_arm           : num  22.5 22.5 22 21.8 21.7 21.5 21.5 21.4 21.4 21.4 ...
##  $ yaw_arm             : num  -161 -161 -161 -161 -161 -161 -161 -161 -161 -161 ...
##  $ total_accel_arm     : int  34 34 34 34 34 34 34 34 34 34 ...
##  $ gyros_arm_x         : num  0 0.02 0.02 0.02 0.02 0.02 0.02 0.02 0.02 0.02 ...
##  $ gyros_arm_y         : num  0 -0.02 -0.03 -0.02 -0.03 -0.03 -0.03 -0.02 0 0 ...
##  $ gyros_arm_z         : num  -0.02 -0.02 0 0 -0.02 0 0 -0.02 -0.03 -0.03 ...
##  $ accel_arm_x         : int  -288 -289 -289 -289 -288 -290 -288 -287 -288 -289 ...
##  $ accel_arm_y         : int  109 110 111 111 109 110 111 111 111 111 ...
##  $ accel_arm_z         : int  -123 -126 -122 -124 -122 -123 -123 -124 -124 -124 ...
##  $ magnet_arm_x        : int  -368 -368 -369 -372 -369 -366 -363 -372 -371 -374 ...
##  $ magnet_arm_y        : int  337 344 342 338 341 339 343 338 331 342 ...
##  $ magnet_arm_z        : int  516 513 513 510 518 509 520 509 523 510 ...
##  $ roll_dumbbell       : num  13.1 12.9 13.4 12.8 13.2 ...
##  $ pitch_dumbbell      : num  -70.5 -70.3 -70.8 -70.3 -70.4 ...
##  $ yaw_dumbbell        : num  -84.9 -85.1 -84.5 -85.1 -84.9 ...
##  $ total_accel_dumbbell: int  37 37 37 37 37 37 37 37 37 37 ...
##  $ gyros_dumbbell_x    : num  0 0 0 0 0 0 0 0 0.02 0 ...
##  $ gyros_dumbbell_y    : num  -0.02 -0.02 -0.02 -0.02 -0.02 -0.02 -0.02 -0.02 -0.02 -0.02 ...
##  $ gyros_dumbbell_z    : num  0 0 0 0 0 0 0 -0.02 -0.02 0 ...
##  $ accel_dumbbell_x    : int  -234 -232 -234 -234 -232 -233 -233 -234 -234 -234 ...
##  $ accel_dumbbell_y    : int  47 46 48 46 47 47 47 48 48 47 ...
##  $ accel_dumbbell_z    : int  -271 -270 -269 -272 -269 -269 -270 -269 -268 -270 ...
##  $ magnet_dumbbell_x   : int  -559 -561 -558 -555 -549 -564 -554 -552 -554 -554 ...
##  $ magnet_dumbbell_y   : int  293 298 294 300 292 299 291 302 295 294 ...
##  $ magnet_dumbbell_z   : num  -65 -63 -66 -74 -65 -64 -65 -69 -68 -63 ...
##  $ roll_forearm        : num  28.4 28.3 27.9 27.8 27.7 27.6 27.5 27.2 27.2 27.2 ...
##  $ pitch_forearm       : num  -63.9 -63.9 -63.9 -63.8 -63.8 -63.8 -63.8 -63.9 -63.9 -63.9 ...
##  $ yaw_forearm         : num  -153 -152 -152 -152 -152 -152 -152 -151 -151 -151 ...
##  $ total_accel_forearm : int  36 36 36 36 36 36 36 36 36 36 ...
##  $ gyros_forearm_x     : num  0.03 0.03 0.02 0.02 0.03 0.02 0.02 0 0 0 ...
##  $ gyros_forearm_y     : num  0 -0.02 -0.02 -0.02 0 -0.02 0.02 0 -0.02 -0.02 ...
##  $ gyros_forearm_z     : num  -0.02 0 -0.03 0 -0.02 -0.02 -0.03 -0.03 -0.03 -0.02 ...
##  $ accel_forearm_x     : int  192 196 193 193 193 193 191 193 193 192 ...
##  $ accel_forearm_y     : int  203 204 203 205 204 205 203 205 202 201 ...
##  $ accel_forearm_z     : int  -215 -213 -215 -213 -214 -214 -215 -215 -214 -214 ...
##  $ magnet_forearm_x    : int  -17 -18 -9 -9 -16 -17 -11 -15 -14 -16 ...
##  $ magnet_forearm_y    : num  654 658 660 660 653 657 657 655 659 656 ...
##  $ magnet_forearm_z    : num  476 469 478 474 476 465 478 472 478 472 ...
##  $ classe              : Factor w/ 5 levels "A","B","C","D",..: 1 1 1 1 1 1 1 1 1 1 ...
\end{verbatim}

\begin{Shaded}
\begin{Highlighting}[]
\NormalTok{clean1 <-}\StringTok{ }\KeywordTok{colnames}\NormalTok{(subprojTraining)}
\NormalTok{clean2 <-}\StringTok{ }\KeywordTok{colnames}\NormalTok{(subprojTraining[, -}\DecValTok{58}\NormalTok{]) }\CommentTok{# Remove the classe column }
\NormalTok{projTesting <-}\StringTok{ }\NormalTok{projTesting[clean1]; projTraining <-}\StringTok{ }\NormalTok{subprojTraining}
\NormalTok{testing <-}\StringTok{ }\NormalTok{testingdata[clean2]}

\KeywordTok{dim}\NormalTok{(projTesting); }\KeywordTok{dim}\NormalTok{(testing)}
\end{Highlighting}
\end{Shaded}

\begin{verbatim}
## [1] 7846   58
\end{verbatim}

\begin{verbatim}
## [1] 20 57
\end{verbatim}

\subsection{Model Builinding \textasciitilde{} Train model with random
forest due to its highly accuracy
rate.}\label{model-builinding-train-model-with-random-forest-due-to-its-highly-accuracy-rate.}

\begin{Shaded}
\begin{Highlighting}[]
\NormalTok{modFitB1 <-}\StringTok{ }\KeywordTok{randomForest}\NormalTok{(classe ~. , }\DataTypeTok{data=}\NormalTok{subprojTraining)}
\NormalTok{predictionsB1 <-}\StringTok{ }\KeywordTok{predict}\NormalTok{(modFitB1, projTesting, }\DataTypeTok{type =} \StringTok{"class"}\NormalTok{)}
\NormalTok{confMatrix <-}\StringTok{ }\KeywordTok{confusionMatrix}\NormalTok{(predictionsB1, projTesting$classe)}
\NormalTok{confMatrix }
\end{Highlighting}
\end{Shaded}

\begin{verbatim}
## Confusion Matrix and Statistics
## 
##           Reference
## Prediction    A    B    C    D    E
##          A 2232    1    0    0    0
##          B    0 1516    6    0    0
##          C    0    1 1362    5    0
##          D    0    0    0 1281    2
##          E    0    0    0    0 1440
## 
## Overall Statistics
##                                           
##                Accuracy : 0.9981          
##                  95% CI : (0.9968, 0.9989)
##     No Information Rate : 0.2845          
##     P-Value [Acc > NIR] : < 2.2e-16       
##                                           
##                   Kappa : 0.9976          
##  Mcnemar's Test P-Value : NA              
## 
## Statistics by Class:
## 
##                      Class: A Class: B Class: C Class: D Class: E
## Sensitivity            1.0000   0.9987   0.9956   0.9961   0.9986
## Specificity            0.9998   0.9991   0.9991   0.9997   1.0000
## Pos Pred Value         0.9996   0.9961   0.9956   0.9984   1.0000
## Neg Pred Value         1.0000   0.9997   0.9991   0.9992   0.9997
## Prevalence             0.2845   0.1935   0.1744   0.1639   0.1838
## Detection Rate         0.2845   0.1932   0.1736   0.1633   0.1835
## Detection Prevalence   0.2846   0.1940   0.1744   0.1635   0.1835
## Balanced Accuracy      0.9999   0.9989   0.9973   0.9979   0.9993
\end{verbatim}

\subsubsection{Let's have a look at the
accuracy}\label{lets-have-a-look-at-the-accuracy}

\begin{Shaded}
\begin{Highlighting}[]
\NormalTok{confMatrix$overall[}\DecValTok{1}\NormalTok{] }\CommentTok{# Accuracy - 0.9980882}
\end{Highlighting}
\end{Shaded}

\begin{verbatim}
##  Accuracy 
## 0.9980882
\end{verbatim}

\subsubsection{It looks very good, it is more then 99.85\%. Random
Forests yielded better Results, as
expected!}\label{it-looks-very-good-it-is-more-then-99.85.-random-forests-yielded-better-results-as-expected}

\section{Conclusion}\label{conclusion}

The estimate the out of sample error is less than 1\% (1 - accuracy).
This is a promising result to detect exercise form to quantify how much
of a particular activity they do and effective.

\end{document}
